\documentclass[11pt,a4paper,parskip]{scrartcl}

\usepackage[ngerman]{babel}
\usepackage[utf8]{inputenc}
\usepackage[T1]{fontenc}
\usepackage{booktabs}

\begin{document}
%
\section{Das Setzen von Zahlen und Einheiten in \LaTeX}
%
% So sehe ich das sehr häufig (falsch gestezte Einiheiten)
Eine Marathon Strecke beträgt 42.2 km. Ein Bekannter hat letzten Monat 100000 Euro im Lotto gewonnen.

% so sollte es aussehen. Standard-LaTeX bietet nur umständliche
% Möglichkeiten, Zahlen und Einheiten korrekt zu setzen:
Eine Marathon Strecke beträgt 42{,}2\,km. Ein Bekannter hat letzten Monat 100\,000 Euro im Lotto gewonnen.
%
\begin{center}
  \begin{tabular}{lcc}
    \toprule
    & falsch & richtig \\
    \midrule
    Ganzzahlen        & $100000000$        & $100\,000\,000$ \\
    Dezimalzahlen (1) & $1.5$              & $1{,}5$ \\
    Dezimalzahlen (2) & $1,5$              & $1{,}5$ \\
    Exponentialzahlen & $1.0\times 10^{2}$  & $1{,}0\cdot 10^{2}$ \\
    Einheiten (1)     & $1 mg$             & $1$\,mg \\
    Einheiten (2)     & $1$\,gHz           & $1$\,GHz \\
    \bottomrule
  \end{tabular}
\end{center}
%
\end{document}
