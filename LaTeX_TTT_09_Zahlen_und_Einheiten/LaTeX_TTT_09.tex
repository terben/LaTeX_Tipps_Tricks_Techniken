\documentclass[11pt,a4paper,parskip]{scrartcl}

\usepackage[ngerman]{babel}
\usepackage[utf8]{inputenc}
\usepackage[T1]{fontenc}

% Das Paket siunitx kümmert sich um das korrekte Setzen von Zahlen
% und Einheiten.
\usepackage{siunitx}

\sisetup{locale = DE,  % Zahlen und Einheiten nach deutschen Regeln
                       % setzen. 'locale = US' würde Regeln aus dem
                       % US-Amerikanischen nutzen.
         per-mode = fraction,  % Einheiten in Bruchschreibweise setzen
         separate-uncertainty} % Zahlen mit Fehler mit '+-' setzen
                               % (Standard ist Klammerschreibweise)

% Aliasse für eigene EInheiten können definiert werden. Nützlich,
% wenn eine bestimmte Einheit öfter zu setzen ist.
\DeclareSIUnit{\ms}{\meter\per\second}

\begin{document}
%
\section{Zahlen und Einheiten}
%
\subsection{Zahlen (\texttt{num}-Kommando)}
0{,}9; 150\,000; $1{,}2\cdot 10^{9}$

\num{0,9}; \num{150000}; \num{1,2e9}

% Der Num-Befehl nimmt Zahlen in 'allen' vernünftigen, auf einem
% Computer üblichen Schreibweisen an und setzt sie korrekt:
\num{0.9}; \num{.9}; \num{1.2d9}

%
\subsection{Einheiten (\texttt{unit}-Kommando)}
m; mN; m\,N; $\frac{\text{m}}{\text{s}}$

% Ich rate Ihnen dringend, für Einheiten die Makro-Schreibweise
% zu verwenden, auch wenn sie manchmal etwas Tipparbeit bedeutet.
% Nur so sind Sie sicher, immer korrekt gesetzte Einheiten zu haben!
\unit{\meter}; \unit{\milli\newton}; \unit{\meter\newton};
\unit{\meter\per\second}

% Möchte man eine Einheit einmalig in einer besonderen Form drucken,
% kamm man Optionen in eckigen Klammern setzen. Ansonsten setzt man
% Optionen am besten global in der Präambel (sisetup-befehl oben)
\unit[per-mode = fraction]{\meter\per\second}

% setze Einheit für die ein Makro in der Präambel definiert wurde.
\unit{\ms}

%
\subsection{Zahlen mit Einheiten (\texttt{qty}-Kommando)
}
1{,}3\,km; $(1{,}0 \pm 0{,}5)$\,m

\qty{1.3}{\kilo\meter};

% Zahlen mit Fehler:
\qty{1.0 +- 0.5}{\meter}
\qty[separate-uncertainty]{1.0 +- 0.5}{\meter}

\subsection{Keine Probleme mehr mit Zahlen und Einheiten}
Eine Marathon Strecke beträgt \qty{42.2}{\kilo\meter}.
Ein Bekannter hat letzten Monat $\num{100000}$ Euro im Lotto gewonnen.

\end{document}
