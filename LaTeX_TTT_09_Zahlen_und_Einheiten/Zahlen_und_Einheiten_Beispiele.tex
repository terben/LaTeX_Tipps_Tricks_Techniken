\documentclass[12pt, a4paper]{scrartcl}

\usepackage[ngerman]{babel}
\usepackage{csquotes}
\usepackage[utf8]{inputenc}
\usepackage[T1]{fontenc}
\usepackage{longtable}
\usepackage{booktabs}
\usepackage{showexpl}
\usepackage{hyperref}
\newcommand{\siunitx}{\texttt{siunitx}}

\lstset{
  numbers=none
}

\usepackage{siunitx}

% globale Einstellungen für siunitx
\sisetup{locale = DE,  % deutsche Darstellungen (z.B. Dezimalkomma
                       % anstatt Dezimalpunkt)
  range-phrase=--,     % Zahlenbereiche werden durch einen Strich dargestellt
  separate-uncertainty % Fehlerangaben von Zahlen werden durch +/-
                       % dargestellt (anstatt in Klammern)
}

% Das Angstrom ist als Einheit nicht in siunitx definiert:
\DeclareSIUnit{\angstrom}{\text {Å}}

% für oft benutzte Einheiten lässt sich ein Alias definieren:
\DeclareSIUnit{\kms}{\kilo\meter\per\second}

\begin{document}
%
\section{\siunitx-Beispiele für Zahlen und Einheiten}
%
Tabelle~\ref{tab:numb_units} zeigt praktische Beispiele für die
Darstellung von Zahlen und Einheiten mit dem \siunitx-Paket.

Für eine ausführliche Darstellung von \siunitx{} schauen Sie bitte in
die sehr gute
\href{http://mirrors.ctan.org/macros/latex/contrib/siunitx/siunitx.pdf}{Anleitung
  des Pakets}.
%
\begin{longtable}{lcp{4cm}}
  \captionabove{Praktische \siunitx-Beispiele}
  \label{tab:numb_units} \\
  %
  \toprule
  \textbf{\LaTeX-Quelltext} & \textbf{Textausgabe} & \textbf{Kommentar} \\
  \midrule
  \endfirsthead
  %
  \toprule
  \textbf{\LaTeX-Quelltext} & \textbf{Textausgabe} & \textbf{Kommentar} \\
  \midrule
  \endhead
  %
  \midrule
  \multicolumn{3}{c}{Fortsetzung folgt} \\
  \bottomrule
  \endfoot
  %
  \bottomrule
  \endlastfoot
  %
  \verb|\qty{8}{\kilo\meter}| & \qty{8}{\kilo\meter} & Einfache Einheit \\
  \verb|\qty{8}{\kilo\meter\per\second}| &
    \SI{8}{\kilo\meter\per\second} & Zusammengesetzte Einheit \\
  \verb|\qty{8.0e03}{\meter\second}| &
    \qty{8.0e03}{\meter\second} & Zahlen können auf viele
    verschiedene Arten angegeben werden (z.~B. Ausgabe von Programmen)\\
  \verb|\qty{0,05}{\milli\second}| &
    \qty{0,05}{\milli\second} & Beachte die unterschiedlichen
    Darstellungen der Einheiten
    \verb|\meter\second| (oben) und hier
    \verb|\milli\second|. Das erste Beispiel benötigt einen kleinen
    Abstand zwischen \texttt{m} und \texttt{s}! \\
  \verb|\qty{10,31}{\meter\tothe{5}}| &
    \qty{10,31}{\meter\tothe{5}} & höhere Potenzen bei den Einheiten \\
  \verb|\qty{245.6 +- 10}{\meter\squared}| &
    \qty{245.6 +- 10}{\meter\squared} & Zahlen mit Fehlerangabe \\
  \midrule
  \verb|\ang{10}| & \ang{10} & Winkel \\
  \verb|\ang{10;5}| & \ang{10;5} & genauere Winkel \\
  \verb|\ang{10;5;2}| & \ang{10;5;2} & noch genauere Winkel \\
  \verb|\num{-9}| & \num{-9} & Negative Zahlen; korrekt
    gesetztes Minuszeichen \\
  \verb|\num{123456789}| &
    \num{123456789} & lange Ganzzahlen (dargestellt in Dreierblöcken) \\
  \verb|\complexnum{1 + 2i}| &
    \complexnum{1+2i} & komplexe Zahlen (korrekte Darstellung der
    imaginären Einheit) \\
  \verb|\num[round-mode=places,%| & & \\
    \verb|     round-precision=4]{12,123456}| &
    \num[round-mode=places,round-precision=4]{12,123456} & Zahl auf 4
    Dezimalstellen gerundet  \\
  \verb|\num[round-mode=figures,%| & & \\
    \verb|     round-precision=4]{12,123456}| &
    \num[round-mode=figures,round-precision=4]{12,123456} & Zahl auf 4
    signifikante Ziffern gerundet  \\
  \midrule
  \verb|\unit{\tera\electronvolt}| &
    \unit{\tera\electronvolt} & Einheiten ohne Zahlen\\
  \verb|\unit[per-mode = fraction]%| & & \\
    \verb|  {\meter\per\second}| &
    \unit[per-mode = fraction]{\meter\per\second} & Einheiten als
    Bruch darstellen\\
  \midrule
  \verb|\qtyrange{2}{10}{\percent}| &
    \qtyrange{2}{10}{\percent}  & Zahlenbereiche;
    korrekt gesetztes Strichsymbol\\
  \midrule
  \verb|\qtylist{2;9;10}{\cm}| &
    \qtylist{2;9;10}{\centi\meter} & Zahlenlisten  \\
  \bottomrule
\end{longtable}
%
\section{Unterstützung für Zahlenspalten in Tabellen}
\siunitx{} unterstützt das Setzen von Tabellenspalten mit Zahlen, die
nach dem Dezimaltrenner ausgerichtet sind. Hierzu ist der
Spaltenausrichtungsparameter \enquote{\texttt{S}} in der
Tabellenpräambel zu nutzen. Es ist auch möglich, Zahlen in der Tabelle
zu runden oder auf eine bestimme Anzahl an Dezimalstellen zu
bringen. Siehe das folgende Beispiel (\LaTeX-Quelltext links und
Textausgabe rechts):
%
\begin{LTXexample}[pos=r,varwidth=true, numbers=none, rframe=]
\begin{tabular}{cSS[round-mode=places,%
                      round-precision=4]}
  \toprule
  \textbf{c-Spalte} &
    \textbf{S-Spalte} &
    \textbf{S-Spalte} \\
   & & {gerundete Zahlen} \\
  \midrule
  1,2345 & 1,2345 & 1,2345 \\
  12,345 & 12,345 & 12,345 \\
  123,45 & 123,45 & 123,45 \\
  1234,5 & 1234,5 & 1234,5 \\
  \bottomrule
\end{tabular}
\end{LTXexample}

\end{document}
