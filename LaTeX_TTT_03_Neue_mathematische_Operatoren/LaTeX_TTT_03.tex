% Kommentierte Quelldatei zu Video 03 der LaTeX Tipps-Tricks-Techniken
% Serie von Thomas Erben
% (siehe https://youtu.be/qlvRS5T9Q-4)

\documentclass[12pt,a4paper]{scrartcl}

\usepackage[ngerman]{babel}
\usepackage[utf8]{inputenc}
\usepackage[T1]{fontenc}
\usepackage{csquotes}

% Pakete für Mathemaik:
\usepackage{amsmath}
\usepackage{amssymb}

% Neue mathematische Operatoren sollten immer
% mit DeclareMathOperator (aus dem amsmath-Paket)
% definiert werden:
\DeclareMathOperator{\ggT}{ggT}

\begin{document}
%
\section{Ein wenig Mathematik}	
Eine bekannte mathematische Beziehung ist:
%
\[
  \sin(x)^{2} + \cos(x)^{2} = 1.
\]
Das können wir auch so schreiben:
%
\[
  \sin^{2} x + \cos^{2} x = 1.
\]
Ein anderes Beispiel ist:
%
\[
  \gcd(21, 3) = 3.
\]
\enquote{$\gcd$} steht hierbei für \emph{greatest common divisor}.
Analog zu oben können wir das auch so schreiben:
%
\[
  \gcd 21, 3 = 3.
\]
Im Deutschen würden wir gerne statt \enquote{$\gcd$} den größten
gemeinsamen Teiler mit dem Operator \enquote{ggT} kennzeichnen:
%
% Achten Sie auf die korrekte Schriftart des ggT-Operators und
% auch korrekter Abstände zwischen Operator und Argumenten. Hierum
% kümmert sich ein mit DeclareMathOperator definierter Operator
% von alleine.
\[
  \ggT(21, 3) = 3 \text{ oder so } \ggT 21, 3 = 3.
\]
\end{document}
