% Die Absatzauszeichung kann in LaTeX für die Koma-Skript
% Klassen mit der Klassenoption 'parskip' von Absatzeinzug auf
% Absatzabstand geändert werden.
\documentclass[a4paper, parskip=half]{scrartcl}
%
\usepackage[utf8]{inputenc}
\usepackage[T1]{fontenc}

% Für Klassen, die die Klassenoption 'parskip' nicht unterstützen
% (z.B. die auf englische Bedürfnisse optimierte Artikelklasse
% 'article') müssen zur Änderung der Absatzauszeichnug das Paket
% 'parskip' einbinden:
% \usepackage{parskip}

\usepackage{amsmath}

% blindtext ist eine Klasse, um Dummy-Text zu erzeugen:
\usepackage{blindtext}

\begin{document}
%
\section*{Absatzauszeichnung in \LaTeX}
%
\LaTeX{} bietet die Möglichkeit eines \emph{Absatzeinzugs} (Standard bei vielen Klassen) oder eines Absatzabstandes (zum Beispiel bei den Koma-Skript Klassen über das Paket über die Klassenoption \texttt{parskip}).

Dies ist der zweite Absatz unseres kleinen Beispieltextes. Danach einer folgen noch weitere \emph{Dummy-Text} Absätze.

\blindtext

\blindtext

\blindtext

\end{document}
