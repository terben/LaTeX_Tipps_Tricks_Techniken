% Erweiterte Quelldatei zu Video 04 der LaTeX Tipps-Tricks-Techniken
% Serie von Thomas Erben (siehe )

\documentclass[12pt,a4paper]{scrartcl}

\usepackage[ngerman]{babel}
\usepackage[utf8]{inputenc}
\usepackage[T1]{fontenc}
\usepackage{csquotes}

\usepackage{hyperref}

\begin{document}
%
\section{Fußnoten}
Hier ein kurzer Beispieltext mit drei Fußnoten: Unter dem
Betriebssystem \textit{Microsoft Windows} gibt es zahlreiche
Webbrowser.\footnote{siehe auch
  \url{https://de.wikipedia.org/wiki/Webbrowser}} Bekannte Vertreter
sind \textit{Firefox}\footnote{siehe auch
  \url{https://www.mozilla.org/de/firefox/new/}} und
\textit{Opera}\footnote{siehe auch \url{http://www.opera.com/de}}.

Sie sollten folgende, typische Fehler beim Setzen von Fußnoten
vermeiden:
%
\begin{enumerate}
  \item Eine Fußnote wird nur dann direkt nach einem Wort gesetzt, wenn
    sich die Fußnote \emph{nur auf dieses eine Wort} beziehen soll.
    Bezieht sich eine Fußnote auf einen gesamten Satz oder auf einen
    Satzteil, so steht die Fußnote \emph{nach dem Satzzeichen}!
  \item Achten Sie darauf, dass direkt vor dem \texttt{\textbackslash
    footnote}-Befehl \emph{keine} Leerzeichen oder Zeilenumbrüche im
    Quelltext stehen! Dies führt im Ausgabedokument zu unerwünschten
    Leerräumen vor dem Fußnotensymbol!
  \item Vermeiden Sie Leerzeichen direkt nach der öffnenden Klammer
    des \texttt{\textbackslash footnote}-Befehl, also etwas wie
    \texttt{\textbackslash footnote\{ Dies ist eine Fußnote.\}}. Dies
    führt zu unerwünschten Leerräumen in der Fußnote selbst.
\end{enumerate}
Beachten Sie, dass der letzte Punkt im Video nicht erwähnt wurde.
\end{document}
